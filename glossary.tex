% units
\newglossary[unit1]{unit}{unit2}{unit3}{Units}
\glsnoexpandfields
\newglossarystyle{units}{
  \renewenvironment{theglossary}
    {\begin{longtable}{cll}}
    {\end{longtable}}
  
  \renewcommand*{\glossaryheader}{
    \textbf{Symbol} & \textbf{Unit} & \textbf{Description} \\ \hline \tiny \endhead
  }

  \renewcommand*{\glsgroupheading}[1]{}

  \renewcommand*{\glossaryentryfield}[4]{
    \small \glstarget{##1}{##2} & \glsentryuseri{##1} & ##3 \\ \tiny \\
  }
  
  \renewcommand*{\glossarysubentryfield}[6]{
    \glossaryentryfield{##2}{##3}{##4}{##5}{##6}
  }

  \renewcommand*{\glsgroupskip}{}
}

\let\DeclareUSUnit\DeclareSIUnit
\let\US\SI
\let\us\si
\DeclareUSUnit\inch{in}

\newglossaryentry{bit}{
  type=unit,
  name={\si{\bit}},
  description={basic unit of information, one binary digit},
  user1={Bit},
  plural={Bits},
  sort=bit
}
\newglossaryentry{byte}{
  type=unit,
  name={\si{\byte}},
  description={group of eight \glspl{bit}},
  user1={Byte},
  plural={Bytes},
  sort=byte
}
\newglossaryentry{mb}{
  type=unit,
  name={\si{\mega\byte}},
  description={\SI{e6}{\byte}},
  user1={Megabyte},
  sort=mb
}
\newglossaryentry{gb}{
  type=unit,
  name={\si{\giga\byte}},
  description={\SI{e9}{\byte}},
  user1={Gigabyte},
  sort=gb
}
\newglossaryentry{gib}{
  type=unit,
  name={\si{\gibi\byte}},
  description={},
  user1={Gibibyte},
  sort=gib
}
\newglossaryentry{s}{
  type=unit,
  name={\si{\seconds}},
  description={base unit of time},
  user1={Second},
  sort=s
}
\newglossaryentry{hz}{
  type=unit,
  name={\si{\hertz}},
  description={unit of frequency, one cycle per second},
  user1={Megahertz},
  sort=hz
}
\newglossaryentry{mhz}{
  type=unit,
  name={\si{\mhz}},
  description={\SI{e6}{\hz}},
  user1={Megahertz},
  sort=mhz
}
\newglossaryentry{ghz}{
  type=unit,
  name={\si{\ghz}},
  description={\SI{e9}{\hz}},
  user1={Gigahertz},
  sort=ghz
}
\newglossaryentry{volt}{
  type=unit,
  name={\si{\volt}},
  description={derived unit of electrical voltage},
  user1={Volt},
  sort=v
}
\newglossaryentry{amp}{
  type=unit,
  name={\si{\ampere}},
  description={base unit of electrical current},
  user1={Ampere},
  sort=a
}
\newglossaryentry{watt}{
  type=unit,
  name={\si{\watt}},
  description={derived unit of power},
  user1={Watt},
  sort=w
}
\newglossaryentry{meter}{
  type=unit,
  name={\si{\metre}},
  description={base unit of length},
  user1={Meter},
  sort=m
}
\newglossaryentry{millimeter}{
  type=unit,
  name={\si{\milli\metre}},
  description={\SI{e-3}{\metre}},
  user1={Millimeter},
  sort=mm
}
\newglossaryentry{inch}{
  type=unit,
  name={\us{\inch}},
  description={\SI{25.4}{\milli\meter}},
  user1={Inch},
  sort=in
}

% ------------------------------------------------------------------------------
% spacing
% ...
% more spacing
% ...
% and even more spacing
% ...
% ACRONYMS ---------------------------------------------------------------------

\newglossaryentry{usb}{
  type=\acronymtype,
  name={USB},
  description={universal serial bus}
}
\newglossaryentry{os}{
  type=\acronymtype,
  name={OS},
  plural={OSs},
  firstplural={operation systems (OSs)},
  description={operating system}
}
\newglossaryentry{cpu}{
  type=\acronymtype,
  name={CPU},
  first={central processing unit (CPU)},
  description={central processing unit}
}
\newglossaryentry{gpu}{
  type=\acronymtype,
  name={GPU},
  first={graphics processing unit (GPU)},
  description={graphics processing uni}
}
\newglossaryentry{raspi}{
  type=\acronymtype,
  name={RasPi},
  first={Raspberry Pi (RasPi)},
  description={\gls{raspberrypi}}
}
\newglossaryentry{CD}{
  type=\acronymtype,
  name={CD},
  description={compact disc}
}
\newglossaryentry{vm}{
  type=\acronymtype,
  name={VM},
  plural={VMs},
  first={virtual machine (VM)},
  firstplural={virtual machines (VMs)},
  description={virtual machine}
}
\newglossaryentry{eg}{
  type=\acronymtype,
  name={e.g.},
  description={example given}
}
\newglossaryentry{bash}{
  type=\acronymtype,
  name={Bash},
  description={Bourne-again shell}
}
\newglossaryentry{url}{
  type=\acronymtype,
  name={URL},
  description={uniform resource locator}
}
\newglossaryentry{aur}{
  type=\acronymtype,
  name={AUR},
  first={\gls{arch} User Repository (AUR)},
  description={\gls{arch} User Repository}
}
\newglossaryentry{rc}{
  type=\acronymtype,
  name={RC},
  first={Release Candidate (RC)},
  description={\gls{relcan}}
}
\newglossaryentry{gui}{
  type=\acronymtype,
  name={GUI},
  description={graphical user interface}
}
\newglossaryentry{ssh}{
  type=\acronymtype,
  name={SSH},
  description={secure shell}
}
\newglossaryentry{cto}{
  type=\acronymtype,
  name={CTO},
  description={chief technical officer}
}
\newglossaryentry{wep}{
  type=\acronymtype,
  name={WEP},
  first={wired equivalent privacy (WEP))},
  description={wired equivalent privacy}
}
\newglossaryentry{wpa}{
  type=\acronymtype,
  name={WPA},
  first={WiFi protected access (WPA)},
  description={WiFi protected access}
}
\newglossaryentry{wpa2}{
  type=\acronymtype,
  name={WPA2},
  first={WiFi protected access 2 (WPA2)},
  description={\gls{wpa}2}
}
\newglossaryentry{iv}{
  type=\acronymtype,
  name={IV},
  first={initialization vector (IV)},
  description={initialization vector}
}
\newglossaryentry{icv}{
  type=\acronymtype,
  name={ICV},
  first={integrity check value (ICV)},
  description={integrity check value}
}
\newglossaryentry{rc4}{
  type=\acronymtype,
  name={RC4},
  description={\gls{rivest} cipher 4}
}
\newglossaryentry{crc}{
  type=\acronymtype,
  name={CRC},
  first={cyclic redundancy check (CRC)},
  description={cyclic redundancy check}
}
\newglossaryentry{crc32}{
  type=\acronymtype,
  name={CRC32},
  description={\gls{crc} with a resulting \SI{32}{\bit} hash value}
}
\newglossaryentry{ksa}{
  type=\acronymtype,
  name={KSA},
  first={key scheduling algorithm (KSA)},
  description={key scheduling algorithm}
}
\newglossaryentry{prga}{
  type=\acronymtype,
  name={PRGA},
  first={pseudo random generation algorithm (PRGA)},
  plural={PRGAs},
  description={pseudo random generation algorithm}
}
\newglossaryentry{ap}{
  type=\acronymtype,
  name={AP},
  plural={APs},
  first={access point (AP)},
  description={access point}
}
\newglossaryentry{psk}{
  type=\acronymtype,
  name={PSK},
  first={pre shared key (PSK)},
  description={pre shared key}
}
\newglossaryentry{eap}{
  type=\acronymtype,
  name={EAP},
  first={extensible authentication protocol (EAP)},
  description={extensible authentication protocol}
}
\newglossaryentry{tkip}{
  type=\acronymtype,
  name={TKIP},
  first={temporal key integrity protocol (TKIP)},
  description={temporal key integrity protocol}
}
\newglossaryentry{ccmp}{
  type=\acronymtype,
  name={CCMP},
  %first={Counter-Mode \gls{cbc}-\gls{mac} Protocol (CCMP)},
  description={counter-mode cipher block chaining message protocol}
}
\newglossaryentry{cbc}{
  type=\acronymtype,
  name={CBC},
  first={cipher block chaining (CBC)},
  description={cipher block chaining}
}
\newglossaryentry{aes}{
  type=\acronymtype,
  name={AES},
  first={advanced encryption standard (AES)},
  description={advanced encryption standard}
}
\newglossaryentry{radius}{
  type=\acronymtype,
  name={RADIUS},
  first={remote authentication dial-in user service (RADIUS)},
  description={remote authentication dial-in user service}
}
\newglossaryentry{hmac}{
  type=\acronymtype,
  name={HMAC},
  first={hash-based message authentication (HMAC)},
  description={hash-based message authentication}
}
\newglossaryentry{ssid}{
  type=\acronymtype,
  name={SSID},
  plural={SSIDs},
  first={service set identifier (SSID)},
  description={service set identifier}
}
\newglossaryentry{wps}{
  type=\acronymtype,
  name={WPS},
  first={WiFi protected setup (WPS)},
  description={WiFi protected setup}
}
\newglossaryentry{gcc}{
  type=\acronymtype,
  name={GCC},
  first={GNU compiler collection (GCC)},
  description={\gls{gnu} Compiler Collection -- a compiler suite (originally known as \gls{gnu} C Compiler and back then only used for compiling C) for many different languages like C(++), ObjectiveC, Java and more}
}
\newglossaryentry{macad}{
  type=\acronymtype,
  name={MAC},
  first={media access control (MAC)},
  description={media access control -- a unique identifier assigned to network interfaces for communications on the physical network segment}
}
\newglossaryentry{mac}{
  type=\acronymtype,
  name={MAC},
  first={message authentication code (MAC)},
  description={message authentication code}
}
\newglossaryentry{mic}{
  type=\acronymtype,
  name={MIC},
  first={message integrity check (MIC)},
  description={message integrity check}
}
\newglossaryentry{osi}{
  type=\acronymtype,
  name={OSI},
  first={open systems interconnection (OSI)},
  description={open systems interconnection}
}
\newglossaryentry{nfc}{
  type=\acronymtype,
  name={NFC},
  first={near field communication (NFC)},
  description={near field communication}
}
\newglossaryentry{pin}{
  type=\acronymtype,
  name={PIN},
  plural={PINs},
  description={personal identification number}
}
\newglossaryentry{pbkdf2}{
  type=\acronymtype,
  name={PBKDF2},
  first={password-based key derivation function 2 (PBKDF2)},
  description={password-based key derivation function 2}
}
\newglossaryentry{pmk}{
  type=\acronymtype,
  name={PMK},
  first={preshared master key (PMK)},
  description={preshared master key}
}
\newglossaryentry{evm}{
  type=\acronymtype,
  name={EVM},
  first={Ethereum virtual machine (EVM)},
  description={\gls{eth} virtual machine}
}
\newglossaryentry{dht}{
  type=\acronymtype,
  name={DHT},
  plural={DHTs},
  first={distributed hashtable (DHT)},
  firstplural={distributed hashtables (DHTs)},
  description={distributed hashtable}
}
\newglossaryentry{ecdsa}{
  type=\acronymtype,
  name={ECDSA},
  first={elliptic curve digital signature algorithm (ECDSA)},
  description={\gls{ec}\gls{dsa}}
}
\newglossaryentry{ec}{
  type=\acronymtype,
  name={EC},
  first={elliptic curve (EC)},
  description={elliptic-curve [cryptography]}
}
\newglossaryentry{dsa}{
  type=\acronymtype,
  name={DSA},
  first={digital signature algorithm (DSA)},
  description={digital signature algorithm}
}
\newglossaryentry{foss}{
  type=\acronymtype,
  name={FOSS},
  first={free, open-source software (FOSS)},
  description={free, open-source software}
}
\newglossaryentry{gnu}{
  type=\acronymtype,
  name={GNU},
  description={"\gls{gnu_long}'s not \gls{unix}"}
}
\newglossaryentry{api}{
  type=\acronymtype,
  name={API},
  first={application programming interface (API)},
  description={application programming interface}
}
\newglossaryentry{iot}{
  type=\acronymtype,
  name={IoT},
  first={Internet of things (IoT)},
  description={Internet of things}
}
\newglossaryentry{m2m}{
  type=\acronymtype,
  name={M2M},
  first={machine to machine (M2M)},
  description={machine to machine}
}
\newglossaryentry{adb}{
  type=\acronymtype,
  name={ADB},
  first={Android debugging bridge (ADB)},
  description={\gls{android} debugging bridge}
}
\newglossaryentry{sdk}{
  type=\acronymtype,
  name={SDK},
  first={software development kit (SDK)},
  description={software development kit}
}
\newglossaryentry{ndk}{
  type=\acronymtype,
  name={NDK},
  first={native development kit (NDK)},
  description={native development kit}
}
\newglossaryentry{hdmi}{
  type=\acronymtype,
  name={HDMI},
  description={high-definition multimedia interface}
}
\newglossaryentry{ble}{
  type=\acronymtype,
  name={BLE},
  first={Bluetooth low energy (BLE)},
  description={Bluetooth low energy}
}
\newglossaryentry{rfid}{
  type=\acronymtype,
  name={RFID},
  first={radio frequency identification (RFID)},
  description={radio frequency identification}
}
\newglossaryentry{spi}{
  type=\acronymtype,
  name={SPI},
  first={serial peripheral interface (SPI)},
  description={serial peripheral interface}
}
\newglossaryentry{uart}{
  type=\acronymtype,
  name={UART},
  first={universal asynchronous receiver-transmitter (UART)},
  description={universal asynchronous receiver-transmitter}
}
\newglossaryentry{gpio}{
  type=\acronymtype,
  name={GPIO},
  description={general purpose \gls{io}}
}
\newglossaryentry{pwm}{
  type=\acronymtype,
  name={PWM},
  first={pulse width modulation (PWM)},
  description={pulse width modulation}
}
\newglossaryentry{ota}{
  type=\acronymtype,
  name={OTA},
  first={over the air (OTA)},
  description={over the air}
}
\newglossaryentry{soc}{
  type=\acronymtype,
  name={SoC},
  plural={SoCs},
  first={system on a chip (SoC)},
  description={system on a chip}
}
\newglossaryentry{io}{
  type=\acronymtype,
  name={I/O},
  description={input / output}
}
\newglossaryentry{art}{
  type=\acronymtype,
  name={ART},
  first={Android runtime (ART)},
  description={\gls{android} runtime}
}
\newglossaryentry{hal}{
  type=\acronymtype,
  name={HAL},
  first={hardware abstraction layer (HAL)},
  description={hardware abstraction layer}
}
\newglossaryentry{mmu}{
  type=\acronymtype,
  name={MMU},
  first={memory management unit (MMU)},
  description={memory management unit}
}
\newglossaryentry{ram}{
  type=\acronymtype,
  name={RAM},
  description={random access memory}
}
\newglossaryentry{4p4c}{
  type=\acronymtype,
  name={4P4C},
  description={4 position 4 contact, often incorrectly referred to as \gls{rj}9, \gls{rj}10 or \gls{rj}22 -- see \gls{mod_con}}
}
\newglossaryentry{6p2c}{
  type=\acronymtype,
  name={6P2C},
  description={6 position 2 contact, often incorrectly referred to as \gls{rj}11 -- see \gls{mod_con}}
}
\newglossaryentry{6p4c}{
  type=\acronymtype,
  name={6P4C},
  description={6 position 4 contact, often incorrectly referred to as \gls{rj}14 -- see \gls{mod_con}}
}
\newglossaryentry{6p6c}{
  type=\acronymtype,
  name={6P6C},
  description={6 position 6 contact, often incorrectly referred to as \gls{rj}25 -- see \gls{mod_con}}
}
\newglossaryentry{8p8c}{
  type=\acronymtype,
  name={8P8C},
  description={8 position 8 contact, often incorrectly referred to as \gls{rj}45 -- see \gls{mod_con}}
}
\newglossaryentry{rj}{
  type=\acronymtype,
  name={RJ},
  description={registered jack -- see \gls{mod_con}}
}
\newglossaryentry{led}{
  type=\acronymtype,
  name={LED},
  description={light emitting diode}
}
\newglossaryentry{a2dp}{
  type=\acronymtype,
  name={A2DP},
  description={advanced audio distribution profile}
}
\newglossaryentry{tts}{
  type=\acronymtype,
  name={TTS},
  first={text-to-speech (TTS)},
  description={text-to-speech}
}
\newglossaryentry{gprs}{
  type=\acronymtype,
  name={GPRS},
  first={general packet radio service (GPRS)},
  description={general packet radio service}
}
\newglossaryentry{wsu}{
  type=\acronymtype,
  name={WSU},
  first={wireless sensor unit (WSU)},
  description={wireless sensor unit}
}
\newglossaryentry{wiu}{
  type=\acronymtype,
  name={WIU},
  first={wireless information unit (WIU)},
  description={wireless information unit}
}
\newglossaryentry{cots}{
  type=\acronymtype,
  name={COTS},
  first={commercial off-the-shelf (COTS)},
  description={commercial off-the-shelf}
}
\newglossaryentry{lan}{
  type=\acronymtype,
  name={LAN},
  first={local area network (LAN)},
  description={local area network}
}
\newglossaryentry{wan}{
  type=\acronymtype,
  name={WAN},
  first={wide area network (WAN)},
  description={wide area network}
}
\newglossaryentry{lorawan}{
  type=\acronymtype,
  name={LoRaWAN},
  description={long range \gls{wan} -- operating in the \gls{ism} band and using \SI{128}{\bit} \gls{aes} for encryption as well as \gls{mic} for frame integrity}
}
\newglossaryentry{gps}{
  type=\acronymtype,
  name={GPS},
  description={global positioning system}
}
\newglossaryentry{ism}{
  type=\acronymtype,
  name={ISM},
  first={industrial, scientific and medical (ISM)},
  description={industrial, scientific and medical [band]}
}
\newglossaryentry{vwc}{
  type=\acronymtype,
  name={VWC},
  first={volumetric water content (VWC)},
  description={volumetric water content}
}
\newglossaryentry{pv}{
  type=\acronymtype,
  name={PV},
  first={photovoltaic (PV)},
  description={photovoltaic}
}
\newglossaryentry{ar}{
  type=\acronymtype,
  name={AR},
  first={augmented reality (AR)},
  description={augmented reality}
}
\newglossaryentry{pc}{
  type=\acronymtype,
  name={PC},
  description={personal computer}
}
\newglossaryentry{lts}{
  type=\acronymtype,
  name={LTS},
  description={long term support}
}
\newglossaryentry{nema}{
  type=\acronymtype,
  name={NEMA},
  first={National Electric Manufacturers Assiciation (NEMA)},
  description={National Electric Manufacturers Assiciation}
}
\newglossaryentry{ramps}{
  type=\acronymtype,
  name={RAMPS},
  first={RepRap Arduino Mega Pololu Shield (RAMPS)},
  description={RepRap \gls{arduino} Mega Pololu Shield}
}
\newglossaryentry{mqtt}{
  type=\acronymtype,
  name={MQTT},
  first={Message Queuing Telemetry Transport (MQTT)},
  description={Message Queuing Telemetry Transport -- a publish-subscribe-based messaging protocol}
}
\newglossaryentry{rest}{
  type=\acronymtype,
  name={REST},
  first={Representational State Transfer (REST)},
  description={Representational State Transfer -- an architectural style for developing web services}
}
\newglossaryentry{json}{
  type=\acronymtype,
  name={JSON},
  first={\gls{javascript} object notation (JSON)},
  description={\gls{javascript} object notation}
}
\newglossaryentry{dc}{
  type=\acronymtype,
  name={DC},
  description={direct current}
}
\newglossaryentry{ac}{
  type=\acronymtype,
  name={AC},
  description={alternating current}
}
\newglossaryentry{psu}{
  type=\acronymtype,
  name={PSU},
  first={power supply unit (PSU)},
  description={power supply unit}
}
\newglossaryentry{ca}{
  type=\acronymtype,
  name={ca.},
  description={circa}
}
\newglossaryentry{gpg}{
  type=\acronymtype,
  name={GPG},
  first={GNU privacy guard (GPG)},
  description={\gls{gnu} privacy guard -- an independent implementation of the \gls{openpgp} standard}
}
\newglossaryentry{pgp}{
  type=\acronymtype,
  name={PGP},
  first={pretty good privacy (PGP)},
  description={pretty good privacy -- a proprietary encryption solution, with the rights owned by Symantec}
}
\newglossaryentry{ietf}{
  type=\acronymtype,
  name={IETF},
  %first={Internet engineering task force (IETF)},
  description={Internet engineering task force}
}
\newglossaryentry{yaml}{
  type=\acronymtype,
  name={YAML},
  description={\gls{yaml} ain't markup language}
}
\newglossaryentry{http}{
  type=\acronymtype,
  name={HTTP},
  description={hypertext transfer protocol -- an application-level protocol for distributed, collaborative, hypermedia information systems~\cite{fielding1999rfc}}
}
\newglossaryentry{https}{
  type=\acronymtype,
  name={HTTPS},
  description={\gls{http} Secure -- an extension of \gls{http} for secure communication over a computer network an widely used on the Internet, often referred to as \gls{http} over \gls{tls}~\cite{rescorla2000http}}
}
\newglossaryentry{ssl}{
  type=\acronymtype,
  name={SSL},
  first={secure socket layer (SSL)},
  description={secure socket layer -- predecessor of \gls{ssl}, now deprecated by the \gls{ietf}}
}
\newglossaryentry{tls}{
  type=\acronymtype,
  name={TLS},
  first={transport layer security (TLS)},
  description={transport layer security -- a protocol that provides communications privacy over the Internet which allows client/server applications to communicate in a way that is designed to prevent eavesdropping, tampering, or message forgery~\cite{dierks1999rfc}}
}
\newglossaryentry{cert_auth}{
  type=\acronymtype,
  name={CA},
  first={certificate authority (CA)},
  description={certificate authority}
}
\newglossaryentry{pojo}{
  type=\acronymtype,
  name={POJO},
  first={plain old \gls{java} object (POJO)},
  firstplural={plain old \gls{java} objects (POJOs)},
  description={plain old \gls{java} object}
}
\newglossaryentry{dsl}{
  type=\acronymtype,
  name={DSL},
  first={domain-specific language (DSL)},
  description={domain-specific language}
}
\newglossaryentry{ide}{
  type=\acronymtype,
  name={IDE},
  first={integrated development environment (IDE)},
  description={integrated development environment}
}
\newglossaryentry{sql}{
  type=\acronymtype,
  name={SQL},
  first={structured query language (SQL)},
  description={structured query language}
}
\newglossaryentry{nosql}{
  type=\acronymtype,
  name={NoSQL},
  first={No\gls{sql}},
  description={non-relational}
}
\newglossaryentry{cli}{
  type=\acronymtype,
  name={CLI},
  first={command line interface (CLI)},
  description={command line interface}
}
\newglossaryentry{js}{
  type=\acronymtype,
  name={JS},
  first={\gls{javascript} (JS)},
  description={\gls{javascript}}
}
\newglossaryentry{jit}{
  type=\acronymtype,
  name={JIT},
  first={just in time (JIT)},
  description={just in time}
}

% ------------------------------------------------------------------------------
% spacing
% ...
% more spacing
% ...
% and even more spacing
% ...
% GLOSSARY ---------------------------------------------------------------------

\newglossaryentry{gnu_long}{
  name={GNU},
  description={the \gls{gnu} \gls{os} is a complete free software system, upward-compatible with \gls{unix}}
}
\newglossaryentry{kernel}{
  name={Kernel},
  description={a computer program that manages input/output requests from software, and translates them into data processing instructions for the \gls{cpu} and other electronic components of a computer. It is a fundamental part of a modern computer's \gls{os}}
}
\newglossaryentry{arm}{
  name={ARM},
  description={the ARM-architecture is a design for microprocessors developed 1983 by the company Acorn and maintained by ARM Ltd. since 1990}
}
\newglossaryentry{posix}{
  name={POSIX},
  description={defines a standard operating system interface and environment, including a command interpreter, and common utility programs to support applications portability at the source code level. It is intended to be used by both application developers and system implementers\footnote{The Open Group Base Specifications -- \url{http://pubs.opengroup.org/onlinepubs/9699919799}}}
}
\newglossaryentry{linux}{
  name={Linux},
  description={a \gls{unix}-like %and mostly \gls{posix}-compliant
  \gls{os} assembled under the model of \gls{foss} development and distribution. The defining component is the \gls{kernel}}
}
\newglossaryentry{kali}{
  name={Kali \gls{linux}},
  description={a \gls{debian}-derived \gls{linux} distribution designed for digital forensics and penetration testing. It is maintained and funded by Offensive Security Ltd\footnote{About Kali Linux -- \url{https://www.kali.org/about-us}}}
}
\newglossaryentry{debian}{
  name={Debian},
  description={a \gls{unix}-like computer \gls{os} and a \gls{linux} distribution that is composed entirely of \gls{foss}\footnote{Debian -- The Universal Operation System -- \url{https://www.debian.org/index.en.html}}}%The whole project was founded by \gls{murdock}}
}
\newglossaryentry{raspberrypi}{
  name={Raspberry Pi},
  description={a series of credit card sized single-board computers by the Raspberry Pi Foundation with the intention of promoting the teaching of basic computer science in schools and developing countries\footnote{Raspberry Pi FAQs -- \url{https://www.raspberrypi.org/help/faqs}}}
}
\newglossaryentry{torvalds}{
  name={Linus Torvalds},
  description={creator and long time principal developer of the \gls{linux} \gls{kernel}, creator of \gls{git}}
}
\newglossaryentry{murdock}{
  name={Ian Murdock},
  description={founder of the \gls{debian}-project and \gls{cto} of the Linux Foundation}
}
\newglossaryentry{atowns}{
  name={Anthony Towns},
  description={author of \gls{deboot}}
}
\newglossaryentry{thinclient}{
  name={thin-client},
  plural={thin-clients},
  description={a computer or a program that depends heavily on another computer (its server) to fulfill its computational roles}
}
\newglossaryentry{git}{
  name={Git},
  description={distributed revision control system, created by \gls{torvalds}}
}
\newglossaryentry{unix}{
  name={Unix},
  description={a family of multitasking, multiuser computer \glspl{os}}
}
\newglossaryentry{tarball}{
  name={tarball},
  description={the name that describes a group or archive of files that are bundled together usually having the \texttt{.tar} file extension}
}
\newglossaryentry{grml}{
  name={grml},
  description={a bootable live system based on \gls{debian}. It includes a collection of \gls{gnu}/\gls{linux} software especially for system administrators and is especially well suited for administrative tasks like installation, deployment and system rescue\footnote{grml.org - Debian Live system / CD for sysadmins and texttool-users -- \url{https://grml.org}}}
}
\newglossaryentry{knoppix}{
  name={KNOPPIX},
  description={an \gls{os} based on \gls{debian} designed to be run directly from a live \gls{cd} or a USB flash drive, one of the first of its kind for any \gls{os}. It was developed by, and named after, \gls{linux} consultant Klaus Knopper, a German electrical engineer and free software developer}
}
\newglossaryentry{ubuntu}{
  name={Ubuntu},
  description={a \gls{debian} based \gls{linux} \gls{os} and distribution, for \glspl{pc}, including smartphones in later versions, and servers\footnote{About Ubuntu -- \url{http://www.ubuntu.com/about/about-ubuntu}}}
}
\newglossaryentry{arch}{
  name={Arch \gls{linux}},
  description={an independently developed, i686/\gls{x86_64} general purpose \gls{gnu}/\gls{linux} distribution versatile enough to suit any role. Development focuses on simplicity, minimalism, and code elegance\footnote{Arch Linux - ArchWiki -- \url{https://wiki.archlinux.org/index.php/Arch_Linux}}}
}
\newglossaryentry{relcan}{
  name={release candidate},
  description={version of a program that is nearly ready for release but may still have a few bugs; the status between beta version and release version~\cite{wiki_release_candidate}}
}
\newglossaryentry{pfsense}{
  name={pfSense},
  description={an open source software distribution that is based on \gls{freebsd} with its target usecases as \gls{os} for routers}
}
\newglossaryentry{freebsd}{
  name={FreeBSD},
  description={a free, \gls{unix}-like computer \gls{os}. It has similarities to \gls{linux} while it differs in terms of delivering things like device drivers, the \gls{kernel} and its own userland out of the box that for example \gls{linux} doesn't.\footnote{FreeBSD: the other free UNIX family -- \url{http://www.informit.com/articles/article.aspx?p=439601}}}
}
\newglossaryentry{jail}{
  name={jail},
  description={a separated environment underneath an already existing \gls{os} that acts as a sandbox and can be used to lock processes / users into said subsystem and not giving them access to the main \gls{os} and therefore increasing security}
}

\newglossaryentry{rivest}{
  name={Ron Rivest},
  description={inventor of the \gls{rc4}-algorithm}
}
\newglossaryentry{xor}{
  name={XOR},
  description={logical operation, also known as exclusive disjunction}
}
\newglossaryentry{sbox}{
  name={S-Box},
  description={allows for substitution, used for symmetric key algorithms}
}
\newglossaryentry{osimodel}{
  name={\gls{osi} model},
  description={a conceptual model for functions of a telecommunication or computing system without regard to their underlying internal structure and technology}
}
\newglossaryentry{diffhell}{
  name={Diffie-Hellman key exchange},
  description={method of securely exchanging cryptographic keys over a public channel}
}
\newglossaryentry{nonce}{
  name={nonce},
  plural={nonces},
  description={an arbitrary number that may only be used once}
}
\newglossaryentry{eth}{
  name={Ethereum},
  description={a decentralized platform that runs smart contracts: applications that run exactly as programmed without any possibility of downtime, censorship, fraud or third party interference.\footnote{Ethereum -- \url{https://ethereum.org}}}
}
\newglossaryentry{btc}{
  name={Bitcoin},
  description={a worldwide cryptocurrency and digital payment system called the first decentralized digital currency, as the system works without a central repository or single administrator.\footnote{Bitcoin -- \url{https://www.bitcoin.com}}}
}
\newglossaryentry{android}{
  name={Android},
  description={a mobile \gls{os} developed by Google, based on a modified version of the \gls{linux} \gls{kernel} and other \gls{foss} and designed primarily for touchscreen mobile devices such as smartphones and tablets.\footnote{Android -- \url{https://www.android.com}}}
}
\newglossaryentry{i2c}{
  name={I$^2$C},
  description={a multi-master, multi-slave, packet switched, single-ended, serial computer bus}
}
\newglossaryentry{mod_con}{
  name={modular connector},
  description={an electrical connector that was originally designed for use in telephone wiring, but has since been used for many other purposes where the probably most well known applications are for telephone and Ethernet. While often falsely being referred to as registered jacks, the \gls{rj} specifications define the wiring patterns of the jacks, not the physical dimensions or geometry of the connectors, which are actually specified under ISO standard 8877~\cite{iso8877, modular_connector}}
}
\newglossaryentry{zigbee}{
  name={ZigBee},
  description={a low-power, low data rate and close proximity wireless ad hoc network}
}
\newglossaryentry{x86}{
  name={x86},
  description={a family of backward-compatible instruction set architectures}
}
\newglossaryentry{x86_64}{
  name={x86-64},
  description={the \SI{64}{\bit} version of the \gls{x86} instruction set}
}
\newglossaryentry{arduino}{
  name={Arduino},
  description={an open source computer hardware and software company, project, and user community that designs and manufactures single-board microcontrollers and microcontroller kits}
}
\newglossaryentry{nema_17}{
  name={NEMA 17},
  description={a stepper motor with a \SI{1.7 x 1.7}{\inch} faceplate as specified by \gls{nema}}
}
\newglossaryentry{docker}{
  name={Docker},
  description={an open platform the performs \gls{os}-level virtualization and allows building, shipping and running the created application containers}
}
\newglossaryentry{ios}{
  name={iOS},
  description={a mobile \gls{os} created and developed by Apple Inc. exclusively for its hardware}
}
\newglossaryentry{openpgp}{
  name={OpenPGP},
  description={an open-source, \gls{ietf}-approved standard describing encryption technologies interoperable with \gls{pgp}. The term can be used to describe any program that supports the OpenPGP system}
}
\newglossaryentry{ruby}{
  name={Ruby},
  description={a dynamic, open source programming language with a focus on simplicity and productivity\footnote{Ruby -- \url{https://www.ruby-lang.org/en}}}
}
\newglossaryentry{zwave}{
  name={Z-Wave},
  description={a wireless communications protocol used primarily for home automation. It builds a mesh network using low-energy radio waves in the \SI{900}{\mega\hertz} range to communicate from appliance to appliance, allowing for wireless control of the same}
}
\newglossaryentry{python}{
  name={Python},
  description={an interpreted, open-source, high-level programming language for general-purpose programming}
}
\newglossaryentry{java}{
  name={Java},
  description={a general-purpose computer-programming language that is concurrent, class-based, object-oriented, and specifically designed to have as few implementation dependencies as possible}
}
\newglossaryentry{gradle}{
  name={Gradle},
  description={an open-source build automation tool focused on flexibility and performance. The build scripts are written using a \gls{groovy} or \gls{kotlin} \gls{dsl}~\cite{gradle_docs}}
}
\newglossaryentry{groovy}{
  name={Groovy},
  description={a optionally typed and dynamic language, with static-typing and static compilation capabilities, for the \gls{java} platform with an easy to learn syntax~\cite{groovy}}
}
\newglossaryentry{kotlin}{
  name={Kotlin},
  description={a statically typed programming language for modern multiplatform applications, 100\% interoperable with \gls{java} and \gls{android}~\cite{kotlin}}
}
\newglossaryentry{node}{
  name={Node[.js]},
  description={an open-source, cross-platform, asynchronous, event driven \gls{javascript} run-time environment that executes \gls{javascript} code server-side, designed to build scalable network applications~\cite{about_node_js}}
}
\newglossaryentry{javascript}{
  name={JavaScript},
  description={a lightweight interpreted or \gls{jit}-compiled programming language with first-class functions, most well-known as the scripting language for web pages~\cite{about_javascript}}
}

% ------------------------------------------------------------------------------
% spacing
% ...
% more spacing
% ...
% and even more spacing
% ...
% COMMANDS ---------------------------------------------------------------------

\newglossaryentry{lsblk}{
  name={\texttt{lsblk}},
  description={lists information about all or the specified block devices and therefore needs to read the \texttt{sysfs} filesystem to gather information\footnote{lsblk - Linux man page -- \url{http://linux.die.net/man/8/lsblk}}}
}
\newglossaryentry{qemu}{
  name={\texttt{qemu}},
  description={a generic and open source machine emulator and virtualizer\footnote{QEMU - Wiki -- \url{http://wiki.qemu.org/Main_Page}}}
}
\newglossaryentry{ifconfig}{
  name={\texttt{ifconfig}},
  description={allows to configure network interfaces\footnote{ifconfig - Linux man page -- \url{http://linux.die.net/man/8/ifconfig}}}
}
\newglossaryentry{mkdir}{
  name={\texttt{mkdir}},
  description={a command for \gls{unix}-based \glspl{os} to create a directory\footnote{mkdir - Linux man page -- \url{http://linux.die.net/man/1/mkdir}}}
}
\newglossaryentry{cd}{
  name={\texttt{cd}},
  description={a command for \gls{unix}-based \glspl{os} to change the current working directory\footnote{cd - Linux man page -- \url{http://linux.die.net/man/1/cd}}}
}
\newglossaryentry{dd}{
  name={\texttt{dd}},
  description={a tool for \gls{unix}-based \glspl{os} for copying and converting files. It's sometimes referred to as ``disk / data destroyer'' due to the tools capabilities like being able to directly write to a disk regardless of \gls{eg} any present filesystem\footnote{dd - Linux man page -- \url{http://linux.die.net/man/1/dd}}}
}
\newglossaryentry{wget}{
  name={\texttt{wget}},
  description={a command for \gls{unix}-based \glspl{os} for non-interactive download of files from the web supporting HTTP, HTTPS and FTP} % TODO entries for HTTP[S] & FTP
}
\newglossaryentry{unzip}{
  name={\texttt{unzip}},
  description={a command for \gls{unix}-based \glspl{os} to list, test and extract compressed files in a ZIP archive}
}
\newglossaryentry{apt}{
  name={\texttt{apt}},
  description={alternatively \texttt{apt-get}, a set of core tools inside \gls{debian} (based \glspl{os}) for un-/installing and updating applications}
}
\newglossaryentry{dpkg}{
  name={\texttt{dpkg}},
  description={the software at the base of the package management system in \gls{debian}\footnote{dpkg - Linux man page -- \url{http://linux.die.net/man/1/dpkg}}}
}
\newglossaryentry{chroot}{
  name={\texttt{chroot}},
  description={an operation that changes the apparent root directory for the current running process and their children. A program that is run in such a modified environment cannot access files and commands outside that environmental directory tree\footnote{Change root - ArchWiki -- \url{https://wiki.archlinux.org/index.php/Change_root}}}
}
\newglossaryentry{deboot}{
  name={\texttt{debootstrap}},
  description={a tool which allows to install a \gls{debian} base system into a directory of another, already installed system or to create bootable disk images. It doesn't require an installation \gls{cd}, just access to a \gls{debian} repository\footnote{Debootstrap - Debian Wiki -- \url{https://wiki.debian.org/Debootstrap}}}
}
\newglossaryentry{rvm}{
  name={\texttt{rvm}},
  description={\gls{ruby} version manager}
}
\newglossaryentry{curl}{
  name={\texttt{curl}},
  description={c\gls{url} is a computer software project providing a library and command-line tool for transferring data using various protocols}
}
\newglossaryentry{whoami}{
  name={\texttt{whoami}},
  description={a command found in many \glspl{os} that prints the effective username of the current user when invoked}
}
\newglossaryentry{docker_compose}{
  name={\texttt{docker-compose}},
  description={a tool for defining and running multi-container \gls{docker} applications. One can use a \gls{yaml} file to configure an application’s services and can then, with a single command, create and start all the services from the configuration}
}
\newglossaryentry{npm}{
  name={\texttt{npm}},
  description={\gls{node} package manager}
}